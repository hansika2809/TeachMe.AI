\documentclass{article}
\usepackage[utf8]{inputenc}
\usepackage{graphicx}
\usepackage{amsmath}
\usepackage{geometry}
\usepackage{float}
\geometry{a4paper, margin=1in}
\usepackage{booktabs} % For professional tables
\usepackage{hyperref}
\hypersetup{
    colorlinks=true,
    linkcolor=blue,
    filecolor=magenta,      
    urlcolor=cyan,
    pdftitle={AI Agent Architecture: TeachMe.AI},
    pdfauthor={Kshitij Gupta},
}
\usepackage{tikz} % For flowcharts
\usetikzlibrary{shapes.geometric, arrows, positioning}
\usepackage{listings} % For code
\usepackage{inconsolata}
\lstset{
    basicstyle=\small\ttfamily,
    breaklines=true,
    frame=single,
    showstringspaces=false,
}

% --- Document Metadata ---
\title{AI Agent Architecture Document: \protect\\ TeachMe.AI}
\author{Hansika Gupta}
\date{}


\begin{document}

\maketitle

% --- Introduction Section ---
\section{Introduction}

\subsection{Project Mission}
\textbf{TeachMe.AI} is an intelligent agent prototype designed to serve as a modular, reliable, and specialized study companion for students. Its mission is to automate common, high-stakes academic tasks—such as summarization, code explanation, and quiz generation—with a higher degree of accuracy than general-purpose LLMs can provide.

\subsection{Architectural Philosophy & Choices}
The core architecture of \texttt{TeachMe.AI} was designed from the ground up to support task specialization. A "monolithic" agent with a single, massive prompt is prone to task-bleeding, hallucination, and unreliability.

Therefore, we selected a \textbf{Gated Router} (or \textbf{Orchestrator}) architecture. This design is composed of:
\begin{enumerate}
    \item A central \textbf{Router Agent} that classifies user intent.
    \item A set of modular \textbf{Specialized "Worker" Agents} that perform specific tasks (e.g., "Text\_Summarizer", "Math\_Solver", "Code\_explainer").
    
\end{enumerate}

The primary reason for this choice is \textbf{modularity}. This architecture allows any specialized agent to be independently upgraded or replaced. This project leverages this design to integrate a custom-fine-tuned summarization model, demonstrating the system's core hypothesis: specialized agents will always outperform generalized ones on their specific tasks.

\subsection{Document Purpose}
This document details the complete architecture of the \texttt{TeachMe.AI} prototype. It covers all required components, including the individual agents, the models used, the data interaction flow, and the reasoning for key design decisions.


% --- System Overview Section ---
% --- System Overview Section (Simplified) ---
% --- System Overview Section (Compact) ---
% --- System Overview Section (Revised with Rationale) ---
\section{System Overview}
The \texttt{TeachMe.AI} system is a multi-stage pipeline that processes user input through a series of validation, routing, and execution steps. The architecture is intentionally designed as a \textbf{Gated Router} (or \textbf{Orchestrator}) pattern.

\subsection{Architectural Choice \& Rationale}
This "Gated Router" architecture was chosen over a single, monolithic agent for several critical reasons:
\begin{itemize}
    \item \textbf{Modularity \& Scalability:} The design is inherently modular. New capabilities (e.g., a "Wikipedia Search Agent") can be added simply by creating a new agent and adding it to the router's "handoff" list, without modifying any other agent.
    \item \textbf{Task Specialization \& Reliability:} Instead of one massive, complex prompt, each of the 10 worker agents has its own focused set of instructions (e.g., \texttt{Math\_Problem\_Solver}). This specialization makes each agent an expert at its one task, leading to significantly more reliable and accurate outputs.
    \item \textbf{Enables Targeted Fine-Tuning:} This is the most important advantage for this project. The architecture allows us to "plug-and-play" different models for different agents. We can use a fast, cost-effective model (Gemini 2.5 Flash) for simple tasks like routing and a powerful, custom-fine-tuned model (T5 model) for a single, token-heavy task like summarization.
    \item \textbf{Controlled \& Safe:} The "gated" part of the design, which uses meta-agent guardrails, provides intelligent, context-aware safety checks that simple regex filters cannot.
\end{itemize}

\subsection{Interaction Flow}
The high-level data flow, illustrated in Figure~\ref{fig:flowchart}, begins with the user interface, passes through pre-processing and safety guardrails, is routed to a specialized agent, and is finally validated by output guardrails before being returned to the user.

% --- The [H] specifier requires the 'float' package: \usepackage{float} ---
\begin{figure}[htbp] 
\centering
% Define tikz styles (COMPACT)
\tikzstyle{block} = [rectangle, draw, fill=blue!20, 
    text width=8em, text centered, rounded corners, minimum height=3em]
\tikzstyle{io} = [trapezium, trapezium left angle=70, trapezium right angle=110, 
    draw, fill=green!20, text width=7em, text centered, minimum height=3em]
\tikzstyle{guardrail} = [diamond, draw, fill=red!20, 
    text width=7em, text badly centered, minimum height=3em, inner sep=0pt]
\tikzstyle{decision} = [diamond, draw, fill=orange!20, 
    text width=7em, text badly centered, minimum height=3em, inner sep=0pt]
\tikzstyle{line} = [draw, -latex']
\tikzstyle{worker} = [rectangle, draw, fill=blue!10, 
    text width=8em, text centered, rounded corners, minimum height=3em]

% Node distance is reduced for a smaller layout
\begin{tikzpicture}[node distance = 1.3cm and 0.8cm]
    % Place nodes
    \node (input) [io] {User Input \& Files (Chainlit UI)};
    \node (parser) [block, below=of input] {File Parser \& Pre-processor};
    \node (router) [decision, below=of parser] {Router Agent (TeachMe.AI)};
    
    % Xshift is reduced to bring branches closer
    \node (worker) [worker, below left=of router, xshift=-2.5cm] {Specialized Agent (e.g., Summarizer)};
    \node (direct) [worker, below right=of router, xshift=2.5cm] {Direct Answer (General Query)};
    
    % Yshift is reduced to bring node up
    \node (response_agg) [block, below=of router, yshift=-1.2cm, text width=5em] {Generated Response};
    
    \node (output) [io, below=of response_agg] {Final Response (Chainlit UI)};

    % Draw edges
    \path [line] (input) -- (parser);
    \path [line] (parser) -- (router);
    
    % MODIFIED LINES: Added 'sloped' and removed x/y shifts
    \path [line] (router) -- node [above, sloped] {Handoff} (worker);
    \path [line] (router) -- node [above, sloped] {General} (direct);
    
    \path [line] (worker) -- (response_agg);
    \path [line] (direct) -- (response_agg);
    
    \path [line] (response_agg) -- (output);
\end{tikzpicture}
\caption{The \texttt{TeachMe.AI} "Gated Router" Interaction Flow (Compact).}
\label{fig:flowchart}
\end{figure}
% --- Core Components Section ---
\section{Core Components}
The \texttt{TeachMe.AI} system is composed of four primary component types: the central Router, the specialized Worker Agents, the Meta-Agent Guardrails, and the UI/Pre-processing layer.

\subsection{The Router Agent (TeachMe.AI)}
This is the central "brain" of the application. Its purpose is not to answer complex questions directly, but to act as an intelligent \textbf{orchestrator}.
\begin{itemize}
    \item \textbf{Prompting Strategy:} The agent is instructed to analyze the user's prompt (and the chat history) and determine the best course of action. It can either hand off to a specialized agent or answer a simple conversational query directly.
    \item \textbf{Handoffs:} The router has a list of 10 "handoff" agents it can delegate to. This modularity is the key to the system's design.
\end{itemize}

\subsection{The Specialized "Worker" Agents}
These are 6 distinct agents, each with a highly focused prompt that instructs it to perform one task exceptionally well. This separation of concerns prevents "prompt bleeding" and improves reliability. The available agents are listed in Table~\ref{tab:agents}.

\begin{table}[h!]
\centering
\caption{List of Specialized Worker Agents}
\label{tab:agents}
\begin{tabular}{@{}ll@{}}
\toprule
Agent Name & Purpose \\ \midrule
\texttt{Text\_Summarizer} & Summarizes text from input or files \\
\texttt{Concept\_Explainer} & Explains complex concepts simply \\
\texttt{Quiz\_Generator} & Creates quizzes from topics or files \\
\texttt{FlashCard\_Generator\_Agent} & Generates flashcards for active recall \\
\texttt{Code\_Explainer\_Agent} & Explains and debugs source code \\
\texttt{Math\_Problem\_Solver} & Solves math problems step-by-step \\
\bottomrule
\end{tabular}
\end{table}


\subsection{UI and Pre-processing Layer}
\begin{itemize}
    \item \textbf{Frontend:} The application uses \texttt{Chainlit} to provide the chat interface, manage user sessions, and handle file uploads.
    \item \textbf{File Handling:} When a user uploads a file, the \texttt{@cl.on\_message} function parses it *before* any agent is called. It extracts text from \texttt{.pdf}, \texttt{.docx}, \texttt{.txt}, and code files, and appends this text to the user's prompt. This crucial step simplifies the agents, as they only need to process plain text.
\end{itemize}

% --- Models Used Section ---
\section{Models Used}
A key design choice in this architecture is the separation of models. Instead of relying on one "god model," \texttt{TeachMe.AI} uses a mix of models, selecting the best one for each specific job.

\subsection{Generalist Model: Gemini 2.5 Flash}
The majority of the system's "thinking" is powered by Google's Gemini 2.5 Flash model.
\begin{itemize}
    \item \textbf{Usage:} This model is used for all "fast" and general-purpose tasks:
        \begin{itemize}
            \item The \texttt{TeachMe.AI} Router Agent
    
            \item 5 out of the 6 Specialized Worker Agents (all except the Summarizer)
        \end{itemize}
    \item \textbf{Reasoning:} Gemini 2.5 Flash was chosen for its optimal balance of speed, capability, and cost-efficiency. For tasks like routing, classification (guardrails), and short-form answers (e.g., \texttt{Concept\_Explainer}), its low latency provides a responsive user experience.
\end{itemize}

\subsection{Fine Tuned Model: Flan-T5-Base Model.}
This model is the core of the assignment and represents the system's "specialized component".

\begin{itemize}
    \item \textbf{Usage:} This model is used exclusively by the \texttt{Text\_Summarizer\_Agent}. By custom-fine-tuning the \textbf{Flan-T5-Base model} on online available dataset (CNN\_dailymail), the model is able to summarize the daily class lectures.
    \item \textbf{Integration:} The fine-tuned \textbf{Flan-T5-Base model} is hosted as a separate endpoint (on Modal AI). The \texttt{Text\_Summarizer\_Agent} is configured to call this endpoint instead of the default Gemini model for summarizing task.
    \item \textbf{Reasoning:} This was a deliberate economic and performance choice. 
        \begin{enumerate}
            \item \textbf{Cost-Effectiveness:} Summarization is the most token-heavy task in the application. Processing large PDF or DOCX files with a pay-per-token API, especially on a free tier, is not scalable. Using a highly efficient, open-source 8B model is significantly more cost-effective.
            \item \textbf{Task Specialization:} It produces accurate, concise, and reliable summaries.
        \end{enumerate}
\end{itemize}
\end{document}